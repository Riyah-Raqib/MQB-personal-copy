\documentclass[xcolor=x11names,compress]{beamer}

%% General document %%%%%%%%%%%%%%%%%%%%%%%%%%%%%%%%%%
\usepackage{graphicx}
\graphicspath{{graphics/}}
\usepackage{tikz}
\usepackage{color,colortbl}
\usepackage{framed}
\usepackage{textcomp, setspace} %Needed for customization of ``listings''
% package
\usepackage[procnames]{listings} % to display code; don't forget [fragile]
% option after \begin{frame}
\input{inputs/rgb}
\definecolor{shadecolor}{rgb}{1,.9,.3}

\usepackage [autostyle]{csquotes}
\MakeOuterQuote{"}

\lstset{
    backgroundcolor=\color{shadecolor},
    tabsize=4,
    rulecolor=,
    language=python,
        basicstyle=\ttfamily\setstretch{1},
        upquote=true,
        aboveskip={1.5\baselineskip},
        columns=fixed,
        showstringspaces=false,
        extendedchars=true,
        breaklines=true,
        prebreak = \raisebox{0ex}[0ex][0ex]{\ensuremath{\hookleftarrow}},
        frame=single,
        showtabs=false,
        showspaces=false,
        showstringspaces=false,
        identifierstyle=\ttfamily,
        keywordstyle=\color[rgb]{0,0,1},
        commentstyle=\color[rgb]{0.133,0.545,0.133},
        stringstyle=\color[rgb]{0.627,0.126,0.941},
        numbers=left, 
        numberstyle=\tiny, 
        stepnumber=2, 
        numbersep=5pt
}

%%%%%%%%%%%%%%%%%%%%%%%%%%%%%%%%%%%%%%%%%%%%%%%%%%%%%%

%% Beamer Layout %%%%%%%%%%%%%%%%%%%%%%%%%%%%%%%%%%
\usetheme{Madrid}
\usecolortheme{seahorse}
\useoutertheme[subsection=false,shadow]{miniframes}
\useinnertheme{default}
% \usefonttheme{serif}
\usepackage{palatino}

\setbeamerfont{title like}{shape=\scshape}
\setbeamerfont{frametitle}{shape=\scshape, series = \bfseries}
\setbeamertemplate{frametitle}[default][center]
\setbeamertemplate{headline}{} %suppress headline (navigation pane)

\setbeamertemplate{footline}
{
  \leavevmode%
  \hbox{
  \begin{beamercolorbox}[wd=.1\paperwidth,ht=2.2ex,dp=1ex,center]{author in head/foot}
    \usebeamerfont{author in head/foot}\insertshortauthor
  \end{beamercolorbox}
  \begin{beamercolorbox}[wd=.8\paperwidth,ht=2.2ex,dp=1ex,center]{title in head/foot}%
    \usebeamerfont{title in head/foot}\insertshorttitle\hspace*{3em}
  \end{beamercolorbox}}
  \begin{beamercolorbox}[wd=.05\paperwidth,ht=2.2ex,dp=1ex,center]{}
     \insertframenumber{} / \inserttotalframenumber\hspace*{-3ex}
  \end{beamercolorbox}
}

\renewcommand{\(}{\begin{columns}}
\renewcommand{\)}{\end{columns}}
\newcommand{\<}[1]{\begin{column}{#1}}
\renewcommand{\>}{\end{column}}

\def\signed #1{{\leavevmode\unskip\nobreak\hfil\penalty50\hskip2em
  \hbox{}\nobreak\hfil(#1)%
  \parfillskip=0pt \finalhyphendemerits=0 \endgraf}}

\newsavebox\mybox
\newenvironment{aquote}[1]
  {\savebox\mybox{#1}\begin{quote}}
  {\signed{\usebox\mybox}\end{quote}}
%%%%%%%%%%%%%%%%%%%%%%%%%%%%%%%%%%%%%%%%%%%%%%%%%%%%
\title[EnE Modelling]{Mathematical Modelling, Theory, and Model Fitting in Ecology \& Evolution}
\subtitle{\it A MulQuaBio Lecture}
  
%%%%%%%%%%%%%%%%%%%%%%%%%%%%%%%%%%%%%%%%%%%%%%%%%%%%

\begin{document}

%%%%%%%%%%%%%%%%%%%%%%%%%%%%%%%%%%%%%%%%%%%%%%%%%%%%%%

\begin{frame}[plain]
	\titlepage
\end{frame}

%%%%%%%%%%%%%%%%%%%%%%%%%%%%%%%%%%%%%%%%%%%%%%%%%%%%%%
\begin{frame}{Outline}
	\begin{itemize}\setlength{\itemindent}{0em}\itemsep12pt
  
	  \item Modelling - what and why
  
	  \item Types of Models
	  
	  \item How to build 'em
	  
	  \item How to test 'em (AKA Fitting Models to Data)

	  \item Summary and Readings
	  \end{itemize}  
  
  \end{frame}
  
%%%%%%%%%%%%%%%%%%%%%%%%%%%%%%%%%%%%%%%%%%%%%%%%%%%%%%
\begin{frame}{Models and Modelling}

	\begin{center}

		\it What does ``modelling'' mean to you?\\
		\vspace{20pt}
		\pause 
		Caricature of a phenomenon that captures its essence \\ (the model's output reproduces/emulates the phenomenon)
		
	  \end{center}
 
 \end{frame}

 %%%%%%%%%%%%%%%%%%%%%%%%%%%%%%%%%%%%%%%%%%%%%%%%%%%%%%
\begin{frame}{Why use (Mathematical) models?}

	\begin{columns}[c]
	  \column{2.3in}
	  \includegraphics[width=\textwidth]{Reef.jpg}
	  \column{2.3in}
	  \includegraphics[width=\textwidth]{Forest.jpg}
	\end{columns}
  
	\pause
  
	\begin{itemize}[<+->] \itemsep4pt
	
		\item To understand/explain an observed phenomenon
  
		\item To develop accurate predictions of an observed phenomenon in the future    
		
		\item To find out what is important to know in an otherwise complex system (cannot measure and monitor everything!)
	  
	\end{itemize}
  
  \end{frame}
  
%%%%%%%%%%%%%%%%%%%%%%%%%%%%%%%%%%%%%%%%%%%%%%%%%%%%%%

\begin{frame}{How to build 'em?}

	\begin{columns}[c]
	  \column{2.3in}
		\includegraphics[width=\textwidth]{Reef.jpg}
	  \column{2.3in}
		\includegraphics[width=\textwidth]{Forest.jpg}
	  \end{columns}
	
	  \begin{center}
		{\it Essentially, all models are wrong, but some are useful}\\--- 
		{\small George Box (1987) (British Mathematician)}
	  \end{center}
	
	  \pause
	  \begin{itemize}[<+->] \itemsep6pt
		\item ``{\it All models are wrong}'' --- every model is a 
		  simplification of reality (e.g., the frictionless pendulum!)
	
		\item ``{\it But some are useful}'' --- the appropriately simplistic 
		  ones can ({\it sufficiently}) explain and predict phenomena
	  \end{itemize}
	
\end{frame}

%%%%%%%%%%%%%%%%%%%%%%%%%%%%%%%%%%%%%%%%%%%%%%%%%%%%%%
\begin{frame}{Models and Modelling}

	\begin{center}

		\it What is more important to you when it come's to a Model's capabilities - Accuracy or Precision?\\
		\vspace{20pt}
		\pause 
		Accuracy is a more fundamental property
		
	  \end{center}
 
 \end{frame}

%%%%%%%%%%%%%%%%%%%%%%%%%%%%%%%%%%%%%%%%%%%%%%%%%%%%%%
\begin{frame}{Two Types of Models}

	\begin{itemize}[<+->]\itemsep20pt

		\item {\it Mechanistic models} aim to explain the PROCESSES or MECHANISMS underlying PATTERNS or PHENOMENA in empirical data 
		\begin{itemize}
			\item These models have a THEORETICAL BASIS
		\end{itemize}
 
		\item {\it Inferential/Empirical (AKA Phenomenological) models} establish the existence of
		STATISTICALLY SIGNIFICANT, NON-RANDOM PATTERNS or PHENOMENA in empirical
		data
		\begin{itemize}
			\item They make no assumptions about the processes or mechanisms that generate the patterns
			\item That is, these models lack a THEORETICAL BASIS
		\end{itemize}
			
	  \end{itemize}
 
 \end{frame}
 
%%%%%%%%%%%%%%%%%%%%%%%%%%%%%%%%%%%%%%%%%%%%%%%%%%%%%%
\begin{frame}{Mechanistic vs. Inferential model fitting}

\begin{center}
		\includegraphics[width=.7\textwidth]{graphics/L-H_cycle.png}\\
		{\it \tiny source: \url{https://www.cds.caltech.edu/~murray/amwiki/images/8/8f/LHgraph.gif}}
\end{center}
\pause
\begin{itemize}[<+->] \itemsep6pt

	\item {\bf Mechanistic model}: \it The Lynx-Hare Cycle is driven by density-dependent population growth in hares

	\item {\bf Inferential model}: \it The Lynx and Hare Cycles have a significant asynchrony (period shift) of x years

\end{itemize}

 \end{frame}
 
 %%%%%%%%%%%%%%%%%%%%%%%%%%%%%%%%%%%%%%%%%%%%%%%%%%%%%%
\begin{frame}{Mechanistic vs. Inferential model fitting}

	\begin{itemize}[<+->]\itemsep10pt
		\item {\it It's not really one vs. the other}; Both types of models play a role in science (and Biology)
		\item Inferential model-fitting reveals patterns in data that generate HYPOTHESES 
 		\begin{itemize}
			\item These can be tested using further model fitting
			\item Example: {\it Whether} climatic temperature affects the Lynx-Hare cycle (using Generalized Linear Model-fitting)
		\end{itemize} 

		\item Mechanistic model-fitting {\it tries} to validate a model that can {\it mechanistically explain} the observed pattern and generate MORE ACCURATE, MECHANISTIC HYPOTHESES
		\begin{itemize}
			\item Example: {\it How} climatic temperature {\it drives} the Lynx-Hare cycle
		\end{itemize} 
		\item \it Ultimately, successful, EMPIRICALLY-GROUNDED mechanistic models are the best path towards a THEORY in any scientific discipline (including ecology and evolution)  
	  \end{itemize}
 
 
 \end{frame}

  %%%%%%%%%%%%%%%%%%%%%%%%%%%%%%%%%%%%%%%%%%%%%%%%%%%%%%
\begin{frame}{Mechanistic vs. Inferential model fitting}
	 
  \begin{center}
	  \includegraphics[width=\textwidth]{Mechanisms.pdf}
  \end{center} 
 
 \end{frame}

 %%%%%%%%%%%%%%%%%%%%%%%%%%%%%%%%%%%%%%%%%%%%%%%%%%%%%%
\begin{frame}{Mechanistic models in Ecology and Evolution?}

\begin{itemize}[<+->] \itemsep12pt
	\item \it Do most ecological studies perform Inferential or Mechanistic modelling (or model-fitting)?
	
	\item The answer is mostly Inferential \pause --- {\it Why?} \pause
	\begin{itemize}
		\item Partly because we are still establishing the existence of GENERAL PATTERNS/PHENOMENA,
		\item  ... and partly because we are (or are forced to be) interested in FORECASTING rather than EXPLAINING. 
 
	\end{itemize}	  
	
	\item {\it So the big question is, can we FORECAST WITHOUT EXPLAINING?}
	
	\begin{itemize}
		\item For example, disease outbreaks: Do we really need to care about the underlying mechanisms if we can predict a future event using Inferential modelling (e.g., Machine-learning of time series patterns)?
		
	\end{itemize}

\end{itemize}

\end{frame}

%%%%%%%%%%%%%%%%%%%%%%%%%%%%%%%%%%%%%%%%%%%%%%%%%%%%%%
\begin{frame}{What are Mechanisms?}

   \begin{itemize}[<+->] \itemsep10pt
		\item Somewhat subjective! 
		\item For example, the Logistic growth model can be thought of as mechanistic:
		$$ 
			\dfrac{dN}{dt} = r N \left(1-\frac{N}{K}\right)  
		$$ 
				
		\item What is the mechanism? \pause --- Density dependence through scramble competition (Brannstrom \& Sumpter 2005)

		\item If the Logistic growth model and another model with contest
			competition were compared with data --- some would call it
			mechanistic  modelling because one is trying to get at the
			underlying mechanism - scramble vs contest competition
			
		\item But is this REALLY mechanistic? What are $r$ and $K$ really?
   \end{itemize}

\end{frame}

%%%%%%%%%%%%%%%%%%%%%%%%%%%%%%%%%%%%%%%%%%%%%%%%%%%%%%
\begin{frame}{Example of a fundamental mechanism: metabolic rate}

	\begin{itemize}
		\item Proponents of {\it Ecological Metabolic Theory} (AKA ``Metabolic Theory of Ecology'') argue that we have not progressed far enough towards mechanistic modelling because metabolism has been ignored
	\end{itemize}

	\begin{center}
		\includegraphics[width=\textwidth]{Mechanisms.pdf}
	\end{center} 	

\end{frame}

%%%%%%%%%%%%%%%%%%%%%%%%%%%%%%%%%%%%%%%%%%%%%%%%%%%%%%
\begin{frame}{Example of a fundamental mechanism: metabolic rate}

	\begin{itemize}
		\item The mechanistic basis of thermal performance curves (\url{https://youtu.be/6n8fCuDwn74}) 
	\end{itemize}

\begin{columns}[c]
  \column{.4\textwidth}
    \pause
      \includegraphics[width=\textwidth]{graphics/Photobacterium.pdf}
  \pause
  \column{.6\textwidth}
  \scriptsize
    $B = B_0 \boxed {e^{-\frac{E}{kT}}}f(T,T_{pk},E_D)$\\
    \vspace{10pt}
    \raggedright{$T$ = temperature (K)\\
     $k$ = Boltzmann constant (eV K$^{-1}$)}\\
     $E$ = Activation energy (eV)\\
     $T_{pk}$ = Temperature of peak performance\\
	 $E_D$ = Deactivation energy (eV)\\
	 
    {\tiny (J H van't Hoff 1884, S Arrhenius 1889)}
\end{columns}

\begin{itemize}\setlength{\itemindent}{0em}
  \pause
  \item Surely there is more to thermal responses?
    % \begin{itemize}\setlength{\itemindent}{-1em}
    %  \item Oxygen limitation
    %  \item Complexity of metabolic network
    %  \item Hormonal regulation
    % \end{itemize}
    % \pause
  \item \textit{What about alternative models?}\\

\end{itemize}
 
\end{frame}

%%%%%%%%%%%%%%%%%%%%%%%%%%%%%%%%%%%%%%%%%%%%%%%%%%%%%%
\begin{frame}{Modelling, and fitting models to data: What's the big idea?}

\begin{itemize} [<+->] \itemsep8pt

	\item {\it If possible}, use biological knowledge to construct models
	\item See if the models ``agree well'' with data
	\item Whichever model ``agrees best'' is most likely to have the right 
	mechanisms
	\item That's the one that's best for predictions (e.g. population 
	cycles), estimating rates (e.g. population or individual growth rates), etc
	\item Don't use models you already know have the wrong mechanisms just because they are popular!
	\item Inferential/statistical models often perform better than mechanistic ones. \pause {\it Why? \pause --- because they have lest restrictive assumptions}   

\end{itemize}
  
\end{frame}

%%%%%%%%%%%%%%%%%%%%%%%%%%%%%%%%%%%%%%%%%%%%%%%%%%%%%%

\begin{frame}{Building Models}

\begin{itemize}\itemsep10pt

	\item It's an art, takes practice (Levins' paper on the  strategy of model building in biology)
	
	\item Build models one mechanism at a time --- in biology, it means 
	start at the right level of organization! 
	
	\item Always consider an alternative that is more parsimonious, even if 
	it is an Inferential model! 

	\item For example, the Boltzmann-Arrhenius model is a good first try 
	describe and uncover mechanisms underlying individual level ``traits'' that are rates (e.g., fecundity or development rate)  

	\item The next step would be to include species interactions with  
	temperature dependence of individuals (or go in an evolutionary direction)

\end{itemize}   

\end{frame}

%%%%%%%%%%%%%%%%%%%%%%%%%%%%%%%%%%%%%%%%%%%%%%%%%%%%%%
\begin{frame}{Models and Modelling?}

	\begin{center}

		\it Which modelling approach (Mechanistic vs Inferential) is more likely to yield accurate predictions under global change?\\
		\vspace{20pt}
		\pause 
		Mechanisms are the key to accurate {\it prediction}!
		
	  \end{center}
 
 \end{frame}

%%%%%%%%%%%%%%%%%%%%%%%%%%%%%%%%%%%%%%%%%%%%%%%%%%%%%%
\begin{frame}{Building EnE Models}
	 
\begin{center}
	\includegraphics[width=\textwidth]{Mechanisms.pdf}
\end{center} 

\begin{itemize}\itemsep2pt

	\item \it You will learn about (deriving and analysing) key models (and theories) at different levels of organization in this part of the course (starting with metabolism)

\end{itemize}   

\end{frame}

%%%%%%%%%%%%%%%%%%%%%%%%%%%%%%%%%%%%%%%%%%%%%%%%%%%%%%
\begin{frame}{Fitting Models (to data)}

\begin{itemize}\itemsep16pt

	\item Least Squares methods
	\begin{itemize}
		 \item Linear 
		 \item Non-linear
	\end{itemize}
	\item Likelihood-based methods 
	\begin{itemize}
	 \item Maximum Likelihood Estimation (MLE) 
	 \item Bayesian
	\end{itemize}
	\item Machine learning and Artificial intelligence

\end{itemize}

\end{frame}

%%%%%%%%%%%%%%%%%%%%%%%%%%%%%%%%%%%%%%%%%%%%%%%%%%%%%%
\begin{frame}{Fitting Models (to data)}

\begin{itemize} \itemsep12pt
	
\item Linear and non-linear least squares model fitting:  (mathematically /algorithmically simple) approaches, useful in many scenarios in biology 
\begin{itemize}
	\item Many mechanisms in biology are inherently non-linear (i.e., r data are better-explained by a non-linear mathematical model)
\end{itemize} 

\item Bayesian (and MLE) methods: Versatile and powerful when data are limited and your (e.g., mechanistic) models are complex (many parameters). Bayesian  more accurate results than MLE if you have prior info (mechanistic models!). 

\item AI/machine Learning: most versatile and powerful for large amounts of noisy data, but the focus on maximizing ability to discover pattern and predict comes at the cost of mechanistic insights


\end{itemize}

\end{frame}


%%%%%%%%%%%%%%%%%%%%%%%%%%%%%%%%%%%%%%%%%%%%%%%%%%%%%%
\begin{frame}{Summary: Model comparison and selection is the key}

\begin{itemize}[<+->]\itemsep12pt
	\item Ideally, several competing (meaningful, not just null) hypotheses (mathematical models) should be fitted to data and compared using statistical theory 
	\item This is an advance over the traditional ``null hypothesis'' approach in Biology
	\item Necessary for the advancement of Biology from from an observational and axiomatic discipline to one with general theories
	\item Necessary for understanding the mechanisms underlying biological patterns/phenomena
\end{itemize}
 
 \end{frame}
  
% %%%%%%%%%%%%%%%%%%%%%%%%%%%%%%%%%%%%%%%%%%%%%%%%%%%%

% \begin{frame}{Comparing and selecting models}

% \begin{itemize}\itemsep12pt
	
% 	\item It's all about the ``Likelihood'' of a model: \\
% 	the set of parameter values of the	model ($\theta$) given outcomes ($x$),	equals the probability of those observed outcomes given those parameter
% 	values, that is,

% 	$$ \mathcal{L}(\theta |x) = P(x | \theta)$$

% 	\item The easiest thing to do for you is to use information theory
% 	(including AIC and BIC) to compare models. 

% 	\item Both AIC and BIC use the {\it estimated (log-) likelihood of a model}:\\
	
% 	\begin{itemize}
		
% 		\item AIC: $-2 \ln[\mathcal{L}(\theta |x)] + 2p$\\

% 		\item BIC (Schwartz criterion): $-2 ln[\mathcal{L}(\theta |x) ] + p \ln(n)$
		
% 		($n = $ sample size, $p$ = number of free parameters)	
% 	\end{itemize}
	
	
% 	\item The lower the AIC or BIC, the better

% \end{itemize}

% \end{frame}

% %%%%%%%%%%%%%%%%%%%%%%%%%%%%%%%%%%%%%%%%%%%%%%%%%%%%

% \begin{frame}{AIC and BIC}

% 	\begin{itemize}\itemsep6pt
		
% 		\item In models fitted with least squares and normally-distributed errors,
		
% 		 $\ln[\mathcal{L}(\theta |x)] = -\frac{n}{2}\ln\left(\frac{RSS}{n}\right)$

% 		\item Thus 
% 		\begin{align*}
% 			\text{AIC} &= -2 \ln[\mathcal{L}(\theta |x)] + 2p \\
% 			&= n + 2 + n \ln\left(\frac{2 \pi}{n}\right) + n \ln(\text{RSS}) + 2 p			
% 		\end{align*}
		
% 		\item And 
% 		\begin{align*}
% 			\text{BIC} &= -2 ln[\mathcal{L}(\theta |x) ] + p \ln(n) \\ 
% 			&= n + 2 + n \ln\left(\frac{2 \pi}{n}\right) + n \ln(\text{RSS}) + p \ln(n)
% 		\end{align*}

% 		\item  \it The small-sample AIC can also be calculated similarly (see Johnson \& Omland 2004)

% 	\end{itemize}
	
% 	\end{frame}

% %%%%%%%%%%%%%%%%%%%%%%%%%%%%%%%%%%%%%%%%%%%%%%%%%%%%%

% \begin{frame}{Comparing and selecting models}

% This is how you calculate AIC and BIC (using python syntax): 

% \pause
% \begin{itemize}\itemsep10pt
% \small
% 	\item residuals = Observations - Predictions
% 	\item rss = sum(residuals ** 2) 
% 	\item Then, AIC = n + 2 + n * log((2 * pi) / n) + n * log(rss) + 2 * p \\
% 		({\it note n and p!})
% 	\item And BIC = n + 2 + n * log((2 * pi) / n) + n * log(rss) + (log(n)) * (p + 1)

% 	\item For both AIC and BIC, If model {\bf A} has AIC lower by 2-3 or 
% 	more than model {\bf B}, it's better --- Differences of less than 2-3 
% 	don't really matter

% \end{itemize}

%     \pause
% Also note that:
% \begin{itemize}
% \small
% 	\item R$^{2}$ = 1 - (rss$/$tss), where tss is total sum of squares: \\
% tss = sum((Observations - mean(Predictions)) ** 2)\\
% (a useful measure of goodness of fit)

% \end{itemize}

% \end{frame}

% %%%%%%%%%%%%%%%%%%%%%%%%%%%%%%%%%%%%%%%%%%%%%%%%%%%%%

% \begin{frame}{Comparing and selecting models: More stuff}

% \begin{itemize} \itemsep12pt
% 	\item You can also calculate Akaike Weights, which is very useful/important
% 	when comparing $ > 2$ models. These weights can then be used to perform {\it
% 	model averaging}

% 	\item Model selection using the Likelihood-Ratio test (LRT) is another
% 	option when you are comparing 2 models
	
% 	\item Adjusted $R^2$ can be used to get a rigorous ``idea'' about how
% 	alternative models are performing
	
% 	\item Very often, you can do step-wise model simplification, especially in
% 	{\it for linear least squares model fitting}: Start with a complex model and
% 	drop terms till you have found a the most {\it parsimonious} simpler version
% 	of the original model
% 	\begin{itemize}
% 		\item There are ready-made functions in {\tt R} to do this (of course!)
% 	\end{itemize}

% \end{itemize}

% \end{frame}

%%%%%%%%%%%%%%%%%%%%%%%%%%%%%%%%%%%%%%%%%%%%%%%%%%%%%%
\begin{frame}{Readings}

\begin{itemize}\itemsep10pt

\item Levins, R. (1966) The strategy of model building in population biology. Am. Sci. 54, 421--431.  

\item Otto, S.P. and Day, T. (2011) A biologist's guide to mathematical modeling in ecology and evolution. Princeton University Press. (Read Chapters 1-2) 

\item Kingsland, Sharon E. (1995) Modeling Nature. University of Chicago Press. (Read over this term - and beyond!)

% \item Johnson, J. B. \& Omland, K. S. (2004) Model selection in ecology 
% and evolution. Trends Ecol. Evol. 19, 101--108. (coming up later)

\item Additional readings on the MQB git repository ({\tt Modelling} directory)
 
\end{itemize}
\end{frame}

\end{document}
